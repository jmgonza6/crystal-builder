\subsection*{Capabilities}

\begin{DoxyVerb}Crystal Builder - Simple program to build several crystal structures for use in many popular 
molecular dynamics and denisty functional theory packages.  
\end{DoxyVerb}



\begin{DoxyItemize}
\item Bravais lattices\+: \hyperlink{class_cubic}{Cubic}, \hyperlink{class_hexagonal}{Hexagonal}, \hyperlink{class_orthorhombic}{Orthorhombic}, \hyperlink{class_tetragonal}{Tetragonal}
\item Library structures\+: ~\newline
 -\/ 2\+D materials\+: Graphene, h-\/\+B\+N, (M)etal (D)i(\+C)halcogenide-\/2\+H/1\+T (i.\+e. Sn\+S2), (T)ransition (M)etal (D)i(\+C)halcogenide-\/2\+H/1\+T (i.\+e. Mo\+Se2)~\newline
 -\/ Energetic materials (P\+E\+T\+N-\/\+I, T\+A\+T\+B, β-\/\+H\+M\+X)
\item Define a custom lattice by specifying the lattice parameters and angles, the basis positions, and the element types
\item Build and view large super cells (no limit on maximum atoms)
\item 3\+D graphics rendering of the crystals with real time user interaction
\item Display as a molecule or with a periodic unit cell
\item Display bonds between nearest neighbors, uniform cutoff
\item Print output atomic coordinates as fractional or cartesian
\item Output to several popular Denisty Functional Theory and Molecular Dynamics packages\+: ~\newline
 -\/ Materials Studio D\+Mol (.car) ~\newline
 -\/ L\+A\+M\+M\+P\+S (.input) ~\newline
 -\/ V\+A\+S\+P 5.\+x.\+x (P\+O\+S\+C\+A\+R)
\end{DoxyItemize}

N\+O\+T\+E\+: If you do not have the G\+T\+K+-\/2.0 libraries installed and in the usual path, i.\+e. {\ttfamily /usr/lib \& /usr/include} for Linux or as a {\ttfamily Framework} on Mac O\+S X, follow the directions below to install them. 



\subsection*{G\+T\+K+-\/2.0 Libraries}

\subsubsection*{Linux}

a) Fedora 21/\+R\+H\+E\+L 6.\+X/\+Cent\+O\+S 6.\+X Run the following command, \begin{DoxyVerb}    yum install gtk2*
\end{DoxyVerb}


a) Debian/\+Ubuntu Run the following command, \begin{DoxyVerb}    apt-get install libgtk2*
\end{DoxyVerb}


\subsubsection*{Mac O\+S X}

a) Install homebrew and then run the following command, \begin{DoxyVerb}    brew install gtk2
\end{DoxyVerb}


\subsubsection*{Windows}

Not supported 



\subsection*{Install main application}

Simply run the following command in the top-\/level directory, \begin{DoxyVerb}    make install 
\end{DoxyVerb}


If you wish to build just the application and not have it installed, run the following, \begin{DoxyVerb}    make all 
\end{DoxyVerb}


In both cases, the executable will be placed in the {\ttfamily build/} directory located in the top-\/level of this distribution 



\subsection*{Usage}

To use the program, simply execute the following in a terminal, \begin{DoxyVerb}    ./crysb  
\end{DoxyVerb}


Currently, three options are available for using this application\+:

{\bfseries 1}) Define all the relevant parameters in the {\ttfamily Lattice}, {\ttfamily Modify}, and {\ttfamily Output} tabs \begin{DoxyVerb}    To begin, select one of the crystals from the drop down and fill in the relevant parameters presented in 
    the pop-up menu.  Then nvaigate to the modify tab to define a chemical element. If you wish to create a  
    supercell, then fill in any of the three empty boxes.  By default, these are set to 1x1x1, and so any of 
    can of them be left empty.  To finish, navigate to the `Output` tab, and select a format from the dropdown.
    If choosing the `LAMMPS` or `DMol` format, then you must specify a file name, using the `Save as` button,
    for the `VASP` format, this is unnecessary since the file written is named `POSCAR`.
\end{DoxyVerb}


{\bfseries 2}) Select a library structure from the dropdown in the {\ttfamily Lattice} tab. \begin{DoxyVerb}    To begin, select `Library` from the `Crystals` dropdown menu.  If building one of hte 2D materials, some 
    interaction may be necessary.  In the case of graphene, the `a` lattice parameter is required.  For the
    layered (T)MDC structures, the default is to use MoS2, and a lattice parameters of 3.19 Å for all four
    cases.  If you wish to build a different crystal, then define the stoichiometry on the bottom of the 
    `Modify` tab.  The syntax for defining the stoichiometry is the following:

          Type1:n1,Type2:n2 i.e.  Mo:1,S:2 or  Sn:1,Se:2.

    If you wish to create a supercell, then fill in any of the three empty boxes.  By default, these are set 
    to 1x1x1, and so any of can of them be left empty.  To finish, navigate to the `Output` tab, and select 
    a format from the dropdown. If choosing the `LAMMPS` or `DMol` format, then you must specify a file name, 
    using the `Save as` button, for the `VASP` format, this is unnecessary since the file written is named `POSCAR`.
\end{DoxyVerb}


{\bfseries 3}) Define a custom crystal with the {\ttfamily \hyperlink{class_custom}{Custom} crystal} button \begin{DoxyVerb}    To build a custom crystal, click the `Custom crystal` button in the `Lattice` tab.  A pop-up menu containing
    entries for the pertinant data will appear. In this scenario, a,b,c and  α,β,γ, the stoichiometry and
    the basis have no defaults and must be explictly defined.  For the stoichiometry, the syntax is the as above,

        Type1:n1,Type2:n2,Type3:n3,... i.e.  C:3 or Mo:1,S:2 or  C:10,H:16,N:8,O:24.    

    Once the lattice paramters, angles and stoichiometry are defined, click the `Add atom` button to begin
    defining the basis.  To define the coordinates, use the fractional system and enter them as a comma
    separated field, i.e. (0.5, 0.25, 0).  To save the coordinates, hit the enter button after all three coordinates
    are define.  If a mistake is made, hit the `Reset` button below this entry field.  This will reset only the
    basis atoms entered so far, and nothing else. 

    ** NOTE The order of atoms added is relevant here and should follow the 
            stoichiometric relationship defined above.

    If you wish to create a supercell, then fill in any of the three empty boxes.  By default, these are set 
    to 1x1x1, and so any of can of them be left empty.  To finish, navigate to the `Output` tab, and select 
    a format from the dropdown. If choosing the `LAMMPS` or `DMol` format, then you must specify a file name, 
    using the `Save as` button, for the `VASP` format, this is unnecessary since the file written is named `POSCAR`.
\end{DoxyVerb}


Once the crystal has been defined and other relevant parameters are set, you can chose to simply build the crystal and have it saved to your file.

Alternatively, if the {\ttfamily Open G\+L} library was used in building this application, the crystal can be viewed in an interactive window by clicking the {\ttfamily Build+render} tab, see below for keyboard and mouse commands. Upon closing this rendering window the crystal data will be written to the file you specified and the main program will terminate.

\begin{DoxyVerb}    ** NOTE: If building a crystal with > 5000 atoms, rendering is a little slow.  Advise
             to use the build without rendering option.
\end{DoxyVerb}


\subsubsection*{Crystal/scene interaction}

\begin{DoxyVerb}    Mouse events:
      Left mouse            rotate/scale the object
      Right mouse           scene display menu
  
    Keyboard events:
      'A'                   display atoms and unit cell
      'a'                   display atoms only
      'B'                   display bonds, uniform nearest neighbor cutoff
      'b'                   hide bonds
      'C'                   increase nearest neighbor cutoff
      'c'                   decrease nearest neighbor cutoff
      'e'                   exit and save the data from the rendering scene
      'o'                   orthographic projection of the crystal
      'p'                   perspective projection of the crystal
      'r' then Left mouse   rotate the crystal in all 3 dimensions
      'z' then Left mouse   zoom toward/away from the crystal
      '0'                   reset orientation of crystal to looking down z toward xy plane
      '1'                   view the xz "a-c" plane of the crystal
      '2'                   view the yz "b-c" plane of the crystal
      '3'                   rotate crystal about x by -45˚, and then -45˚ about z, 'psuedo' perspective view
      '+'                   increase the particle radius uniformly
      '-'                   decrease the particle radius uniformly
\end{DoxyVerb}
 



\subsection*{Info }

{\bfseries Author\+:} \begin{DoxyVerb}/**
  * Joseph M. Gonzalez
  * PhD Student, Materials Simulation Laboratory
  * Department of Physics, University of South Florida, Tampa, FL 33620
  *
  * web: msl.cas.usf.edu
  */
\end{DoxyVerb}


{\bfseries Date\+:} Sep 17, 2015

{\bfseries email\+:} \href{mailto:jmgonza6@mail.usf.edu}{\tt jmgonza6@mail.\+usf.\+edu} 